\documentclass{article}
\usepackage[utf8]{inputenc}
\usepackage{geometry}
\geometry{margin=1in}
\setlength\parindent{0pt}


\title{Knapsack Dynamic Problem}
\author{KV}
\date{December 2020}

\begin{document}

\maketitle

\section{Decision Epochs}
Decisions are made at the beginning of each month

\section{State Space}
State space is defined by current budget for various PPEs for future month, amount of budget already spent on PPEs, current patient waitlist, and expected monthly demand (?)
\[ \vec{s}  = (\vec{bc}, \vec{bu}, \vec{pw}, \vec{pe})\]
\begin{itemize}
    \item $\vec{bc} = bc_{mp}$ - Expected budget for month $m$, and PPE $p$
    \item $\vec{bu} = bc_{mp}$ - Used budget for month $m$, and PPE $p$
    \item $\vec{pw} = pw_{mdc}$ - Number of patients of difficulty $d$, CPU $c$, on a wait list for $m$ months
    \item $\vec{pe} = pe_{dc}$ - Number of patients of difficulty $d$, CPU $c$ expected to arrive this month (Not sure if this is good, because right now, it is assumed this information is know in advance, while in real world this would be realized over the transition)
\end{itemize}

\section{Action Sets}
At the beginning of each month, decision maker must decide how to use the PPE budget, ie how many patients of each group to assign to this month.
\[ \vec{a} = (\vec{s}) \]
\begin{itemize}
    \item $\vec{s} = s_{tmdc}$ - Number of patients of difficulty $d$, CPU $c$, who have been in wait list for $m$ month, to schedule in month $t$
\end{itemize}
The actions must satisfy the following constraints:
\begin{itemize}
    \item Total allocated budged must not be exceeded ($U_{pdc}$ - usage of PPE $p$ per patient difficulty $d$, CPU $c$)
        \[ \sum_{dc}s_{mdc}U_{pdc} >= (bc - bu)\ \quad \forall m, p \]
\end{itemize}

\section{Transition Probabilities}
There are three sources of uncertainty: 
\begin{enumerate}
    \item Number of patients arriving this month - $pe_{dc}$
    \item Transition between patient difficulties within the wait list - $pw_{mdc}$
    \item Amount of resources used during the operation - $U_{pdc}$
\end{enumerate}

Assume $pe_{dc}$ follows some poisson distribution. \\
Assume transition within $pw_{mdc}$ follows some binomial distribution with a probability that patient of difficulty $d$ in CPU $c$ moves to a more difficult category at each state transition. \\
These two sources of uncertainty only deal with $pe_{dc}$ and $pw_{mdc}$ part of the states and must follow the following constraints
\begin{itemize}
	\item Total number of patients on the wait-list must be consistent over time
	\[ \sum_{d} pw'_{m+1,dc} = \sum_{d} pw_{mdc} - \sum_{td} s_{tmdc}\ \quad \forall m, c \]
	\item Patients can only get more difficult over time
	\[ \sum_{di = d}^{D} pw'_{m+1,di,c} \ge \sum_{di = d}^{D} pw_{mdc} - \sum_{t,d=di}^{D} s_{tmdc} \quad \forall m,d,c \]
	\item  Unscheduled expected arrivals are moved into the wait list
	\[ pw'_{m=1, dc} = pe_{dc} -\sum_{t} s_{t, m=0, dc} \quad \forall dc \]
\end{itemize}
    
\section{Costs}
Cost will come only from one source: the number of people unscheduled after action is performed:
\[ c(\vec{s}, \vec{a}) = \sum_{dc} c_m pw_{mdc} \]
Where $c_m$ is computed as follows (where $val$ is an arbitrary number that would allow us to describe cost growth well):
	\[ c_{m} = val^m \quad \forall m \in \{ 0, ..., M-1 \} \]
	\[ c_{M} = val^m * 100  \]
The reason there is a very high cost at the last month is because patients are essentially removed if they are not scheduled and they are in the last wait month (in order to be able to have a limit on $m$ while modeling). So theoretically, the model should schedule in such a way that people never reach that point.

\end{document}

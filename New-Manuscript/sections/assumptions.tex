\section{Assumptions} 
\begin{enumerate}
	\item Assume arrivals follow poisson distribution and all arrivals are at the beginning of the period	
	
	\item It is assumed that patients do not change complexities/priorities within some limit ($TL$). 	
	\item We assume patient complexity/priority transitions follow binary distribution
	
	\item It is assumed if a patient is scheduled in period one, it means they are served "immediately", regardless if in practice the appointment is at the beginning or end of the period.
	
	\item We distinguish between two types of reschedules: good and bad reschedules
	\begin{itemize}
		\item Good reschedules are reschedules where a patient is rescheduled to an earlier period
		\item Bad reschedules are reschedules where a patient is rescheduled to a later period
	\end{itemize}
	
	\item We assume that only specific reschedules are allowed (to simplify the model, and remove  redundancies)
	\begin{itemize}
		\item Good reschedules are allowed from any period after 1 into period 1
		\item Bad reschedules are only allowed from period 1 to any period after 1
	\end{itemize}
	
	\item It is assumed that there is a certain default expected number of PPE units available for all periods
	\begin{itemize}
		\item However, in the period 1, there is some random deviation from the expected number of units
		\item This random deviation follows some uniform distribution
	\end{itemize}
	\item We allow some violation of PPE units, in rare cases (with high cost), in order to accomodate changes due to variability.
	\begin{itemize}
		\item If we have violation of PPE units - the extra capacity comes from external source
	\end{itemize}
\end{enumerate}